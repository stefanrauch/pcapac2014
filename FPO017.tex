\documentclass[a4paper,
               %boxit,
               %titlepage,   % separate title page
               %refpage      % separate references
              ]{jacow}

\makeatletter%                           % test for XeTeX where the sequence is by default eps-> pdf, jpg, png, pdf, ...
\ifboolexpr{bool{xetex}}                 % and the JACoW template provides JACpic2v3.eps and JACpic2v3.jpg which might generates errors
 {\renewcommand{\Gin@extensions}{.pdf,%
                    .png,.jpg,.bmp,.pict,.tif,.psd,.mac,.sga,.tga,.gif,%
                    .eps,.ps,%
                    }}{}
\makeatother

\ifboolexpr{bool{xetex} or bool{luatex}} % test for XeTeX/LuaTeX
 {}                                      % input encoding is utf8 by default
 {\usepackage[utf8]{inputenc}}           % switch to utf8

\usepackage[USenglish]{babel}


\ifboolexpr{bool{jacowbiblatex}}%        % if BibLaTeX is used
 {%
  \addbibresource{jacow-test.bib}
  \addbibresource{biblatex-examples.bib}
 }{}

\newcommand\SEC[1]{\textbf{\uppercase{#1}}}

%%
%%   Lengths for the spaces in the title
%%   \setlength\titleblockstartskip{..}  %before title, default 3pt
%%   \setlength\titleblockmiddleskip{..} %between title + author, default 1em
%%   \setlength\titleblockendskip{..}    %afterauthor, default 1em

%\copyrightspace %default 1cm. arbitrary size with e.g. \copyrightspace[2cm]

% testing to fill the copyright space
%\usepackage{eso-pic}
%\AddToShipoutPictureFG*{\AtTextLowerLeft{\textcolor{red}{COPYRIGHTSPACE}}}

\begin{document}

\title{Managing Multiple Function Generators for FAIR}

\author{S. Rauch, M. Thieme, Ralph C. Bär, GSI,  Darmstadt, Germany}

\maketitle

%
\begin{abstract}
In the FAIR control system, equipment which needs to be controlled with ramped nominal values (e.g. power converters) is controlled by a standard front-end control unit (SCU). An SCU combines a ComExpressBoard with Intel CPU and an FPGA baseboard and acts as bus-master on the SCU host-bus. Up to 12 function generators can be implemented in slave-board FPGAs and can be controlled from one SCU.

The real-time data supply for the generators demands a special software/hardware approach. Direct control of the generators with a FESA (front-end control software) class, running on an Intel Atom CPU with Linux, does not meet the timing requirements. So an extra layer with an LM32 soft-core CPU is added to the FPGA. Communication between Linux and the LM32 is done via shared memory and a ring buffer data structure. The LM32 supplies the function generators with new parameter sets when it is triggered by interrupts. This two-step approach decouples the Linux CPU from the hard real-time requirements. For synchronous start and coherent clocking of all function generators, special pins on the SCU backplane are being used to avoid bus latencies.\end{abstract}


\begin{figure}[!htb]
   \centering
   \includegraphics*[width=65mm]{JACpic_mc}
   \caption{Layout of papers.}
   \label{l2ea4-f1}
\end{figure}

\subsection{Fonts}

In order to produce good Adobe Acrobat PDF files,
authors using a \LaTeX\ template are asked to use only Times (in roman [standard],
bold or italic) and symbols from  the standard set of fonts. In Word use only Symbol
and, depending on your platform, Times or Times New Roman fonts in standard, bold or
italic form.

\begin{figure*}[!tbh]
    \centering
    \includegraphics*[width=\textwidth]{JACpic2v3}

    \caption{Example of a full-width figure showing the JACoW Team at their annual
             meeting in 2012. This figure has a multi-line caption that has to
             be justified rather than centered.}
    \label{l2ea4-f2}
\end{figure*}





Each figure and table must be numbered in ascending order (1, 2, 3, etc.) throughout
the paper. Figure captions (\SI{10}{pt} font) are placed below figures, and table captions are placed above tables. Single-line captions are centered in the column, while captions that span more than one line should be justified. The \LaTeX\ template uses the ‘booktabs’ package to
format tables.

A simple way to introduce figures into a Word document is to place them inside a table that has no borders. This is done in Word as described below.

\textit{Note: If the figure spans both columns, do all steps. If
the figure is contained in a single column, start at step 5.}

\begin{Enumerate}
\item	Insert a continuous section break.
\item	Insert two empty lines (makes later editing easier).
\item	Insert another continuous section break.
\item	Click between the two section breaks and Page Layout $\rightarrow$
        Columns $\rightarrow$ Single.
\item	Insert $\rightarrow$ Table select a one-column, two-row table.
\item	Paste the figure in the first row of the table and adjust the size as appropriate.
\item	Paste/Type the caption in the second row and apply the appropriate figure caption style.
\item	Table $\rightarrow$ Table properties $\rightarrow$ Borders and
        Shading $\rightarrow$ None.
\item	Table $\rightarrow$ Table properties $\rightarrow$ Alignment $\rightarrow$ Center.
\item	Table $\rightarrow$ Table properties $\rightarrow$ Text wrapping $\rightarrow$ None.
\item	Remove blank lines from in and around the table.
\item	If necessary play with the cell spacing and other parameters to improve appearance.
\end{Enumerate}

\section{styles}




\section{checklist for electronic publication}

\begin{Itemize}
    \item  Use only Times or Times New Roman (standard, bold or italic) and Symbol
           fonts for text---\SI{10}{pt} minimum except references, which can be \SI{9}{pt} or \SI{10}{pt}.
    \item  Figures should use Times or Times New Roman (standard, bold or italic) and
           Symbol fonts when possible---\SI{6}{pt} minimum.
    \item  Check that citations to references appear in sequential order and
           that all references are cited~\cite{exampl-last}.
    \item  Check that the PDF file prints correctly.
    \item  Check that there are no page numbers.
    \item  Check that the margins on the printed version are within \SI{\pm1}{mm}
           of the specifications.
    \item  \LaTeX\ users can check their margins by invoking the
           \texttt{boxit} option.
\end{Itemize}


%
% this setting when the default (\flushend)
% => "balance two column" shows bad results
%
\iftrue   % balancing with bad results
	\newpage
	\raggedend
\fi


\iffalse  % only for "biblatex"
	\newpage
	\printbibliography

% "biblatex" is not used, go the "manual" way
\else


\fi

\end{document}
