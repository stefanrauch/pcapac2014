\documentclass[a4paper,
               %boxit,
               %titlepage,   % separate title page
               %refpage      % separate references
              ]{jacow}

\makeatletter%                           % test for XeTeX where the sequence is by default eps-> pdf, jpg, png, pdf, ...
\ifboolexpr{bool{xetex}}                 % and the JACoW template provides JACpic2v3.eps and JACpic2v3.jpg which might generates errors
 {\renewcommand{\Gin@extensions}{.pdf,%
                    .png,.jpg,.bmp,.pict,.tif,.psd,.mac,.sga,.tga,.gif,%
                    .eps,.ps,%
                    }}{}
\makeatother

\ifboolexpr{bool{xetex} or bool{luatex}} % test for XeTeX/LuaTeX
 {}                                      % input encoding is utf8 by default
 {\usepackage[utf8]{inputenc}}           % switch to utf8

\usepackage[USenglish]{babel}


\ifboolexpr{bool{jacowbiblatex}}%        % if BibLaTeX is used
 {%
  \addbibresource{jacow-test.bib}
  \addbibresource{biblatex-examples.bib}
 }{}

\newcommand\SEC[1]{\textbf{\uppercase{#1}}}

%%
%%   Lengths for the spaces in the title
%%   \setlength\titleblockstartskip{..}  %before title, default 3pt
%%   \setlength\titleblockmiddleskip{..} %between title + author, default 1em
%%   \setlength\titleblockendskip{..}    %afterauthor, default 1em

%\copyrightspace %default 1cm. arbitrary size with e.g. \copyrightspace[2cm]

% testing to fill the copyright space
%\usepackage{eso-pic}
%\AddToShipoutPictureFG*{\AtTextLowerLeft{\textcolor{red}{COPYRIGHTSPACE}}}

\begin{document}

\title{Managing Multiple Function Generators for FAIR}

\author{S. Rauch, M. Thieme, Ralph C. Bär, GSI,  Darmstadt, Germany}

\maketitle

%
\begin{abstract}
In the FAIR control system, equipment which needs to be controlled with ramped nominal values (e.g. power converters) is controlled by a standard front-end control unit (SCU). An SCU combines a ComExpressBoard with Intel CPU and an FPGA baseboard and acts as bus-master on the SCU host-bus. Up to 12 function generators can be implemented in slave-board FPGAs and can be controlled from one SCU.

The real-time data supply for the generators demands a special software/hardware approach. Direct control of the generators with a FESA (front-end control software) class, running on an Intel Atom CPU with Linux, does not meet the timing requirements. So an extra layer with an LM32 soft-core CPU is added to the FPGA. Communication between Linux and the LM32 is done via shared memory and a ring buffer data structure. The LM32 supplies the function generators with new parameter sets when it is triggered by interrupts. This two-step approach decouples the Linux CPU from the hard real-time requirements. For synchronous start and coherent clocking of all function generators, special pins on the SCU backplane are being used to avoid bus latencies.\end{abstract}


\section{Description of SCU and FG}
The quadratic function generator(FG) which is described in this paper, is a VDHL macro that runs in SCU bus slave cards. At the moment, there are three slave cards with this feature: DIOB (1 FG), ADDAC1 (2 FGs) and ADDAC2 (2 FGs). The DIOB card has an digital output for the set value of the FG. The two ADDAC cards offer an analog output for the FGs. The slave cards are
controlled via the scu bus from the Scalable Control Unit(SCU). The SCU is a FPGA based controller equipped with a ComExpress Board which runs linux.
The system should be used as an arbitrary function generator with 12 independend channels. Each channel will control equipment that needs ramped nominal values. That means for example  power converters and DDS systems.
The FG is configured with a set of data and interpolates then a predefined number of output values. After the interpolation is started, the FG waits for the next set of data that
is provided by the linux FESA class. A brief hardware description of the FG can be found here (cite FG quad paper).

\section{data supply with real time boundaries}
The FG can be configured to interpolate in steps from 250 up to 32000. The sample frequency is configurable from 16kHz up to 1Mhz. If the FG should now sample with 1Mhz for 250 steps
that means the linux program has to provide a new data set every 250µs. This data rate is to high for linux to be serviced reliably for 12 channels.

\section{implementation with LM32 and msi}
two step approach for data exchange
softcore microcontroller lm32
no operating system for easy real time behaviour
wishbone dualport ram
ringbuffers
msi
\section{future work}

\begin{Itemize}
  \item interrupt driven communication between linux and lm32
  \item
\end{Itemize}

%
% this setting when the default (\flushend)
% => "balance two column" shows bad results
%
\iftrue   % balancing with bad results
	\newpage
	\raggedend
\fi


\iffalse  % only for "biblatex"
	\newpage
	\printbibliography

% "biblatex" is not used, go the "manual" way
\else


\fi

\end{document}
